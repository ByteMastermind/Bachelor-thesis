\chapter{Conclusion}

The aim of the work was to build an open-source device that would be able to clone and emulate RFID cards and tags. Firstly, designing the hardware and software architecture of a portable device, with respect to modularity and future extensibility using a commercially available microcomputer such as Raspberry Pi. Then creating a program allowing cloning and emulation of at least 3 different tags, with debugging directly in the created device. Associated with this is the development of a user interface that will allow easy control and interaction with the device. The last part of the work was to involve testing of the device with different types of RFID tags and cards and evaluating its functionality.

A prototype of a portable and scalable device was designed using a Raspberry Pi microcomputer and the RFID tool Proxmark, which allows cloning or emulation of a number of the most well-known RFID card and tag types such as Mifare Classic, Mifare Ultralight or Legic Prime. A Python program has been developed with a graphical interface that can operate the device in a simple and intuitive way.

The prototype device was carefully tested on purchased and existing tags and cards. It turned out that it can read, clone or emulate the tested cards and tags. Only two specific types of tags could not be emulated, due to a propable hardware issue of the used RFID module. However, this is compensated by the fact that the device can clone these types without significant issues.

At this stage of development, the device is functional for the most of the given types of RFID technology, but further modifications will be needed to improve some features.
