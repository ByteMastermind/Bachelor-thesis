\chapter{Related Work}

Several additional steps are needed to improve this prototype device to make it more compact, and more user-friendly.

One of these steps is to incorporate some batteries directly into the device. While a power bank makes the device portable, the same cannot be said for compactness. To this would be added the incorporation of some sort of start button to start the device. 

It is definitely worth thinking about a different and better touch screen, because with the current one there were problems related to the refresh rate or even in the actual detection of the user's touch.

The next step should be to create a device box, for example in a 3D printer. This would increase the shatter resistance of the device, the overall appearance of the device, but also the compactness of the device, where the user would not be afraid to grab the device in one hand.

It would also be beneficial to modify the operating system running on the Raspberry Pi more, mainly to reduce boot time. Another modification could be a modification to the Proxmark3 program itself, where it would be possible to disable the emulation of certain tags with a simple command rather than a button on the Proxmark.

The door is also open for creating support for more tag types that Proxmark3 itself can handle, or will be able to handle in the future.

Finally, it is definitely important to test the device properly with a large number of tag types, with more than I had available, to find as many bugs as possible within the device software.