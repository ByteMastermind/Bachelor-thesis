\section{Installation}\label{installation}

\subsection{OS installation}\label{os-installation}

Install RaspianOS Bullseye Legacy 32 bit (released 2024-03-12) full
desktop on the Raspberry Pi, tested on Raspberry Pi 4 model B.

Keep username \texttt{pi} as it is necessary to complete installation of
touch screen drivers successfully.

\subsection{Preparing the OS for the
touchscreen}\label{preparing-the-os-for-the-touchscreen}

Used was Waveshare LCD 35B-v2 touchscreen. For more details in case of
utilization of different hardware parts or operating systems follow
tutorial: https://www.waveshare.com/wiki/3.5inch\_RPi\_LCD\_(A)

Following commands prepare the operating system for the touchscreen by
downloading installing necessary packages and screen driver.

\begin{Shaded}
\begin{Highlighting}[]
\FunctionTok{sudo}\NormalTok{ apt update}
\FunctionTok{sudo}\NormalTok{ apt upgrade}
\FunctionTok{sudo}\NormalTok{ apt install cmake}
\FunctionTok{sudo}\NormalTok{ apt{-}get install libraspberrypi{-}dev raspberrypi{-}kernel{-}headers}
\FunctionTok{rm} \AttributeTok{{-}rf}\NormalTok{ Bookshelf Documents/ Downloads/ Music/ Pictures/ Public/ Templates/ Videos/}
\FunctionTok{git}\NormalTok{ clone https://github.com/waveshare/LCD{-}show.git}
\BuiltInTok{cd}\NormalTok{ LCD{-}show/}
\FunctionTok{chmod}\NormalTok{ +x LCD35B{-}show{-}V2}
\ExtensionTok{./LCD35B{-}show{-}V2}
\BuiltInTok{cd}\NormalTok{ \textasciitilde{}}
\end{Highlighting}
\end{Shaded}

\subsubsection{Removing cursor}\label{removing-cursor}

To remove the cursor in the OS, simply add a \texttt{nocursor} option as
follows in the file (\texttt{/etc/lightdm/lightdm.conf})

\begin{verbatim}
xserver-command = X -nocursor
\end{verbatim}

No cursor is displayed whatsoever. You can still put your finger on the
touch screen and do what you normally do with your mouse pointer,
clicking or dragging.

\subsection{Installing Proxmark
software}\label{installing-proxmark-software}

Following command installs necessary dependencies for the Proxmark3
software.

\begin{verbatim}
sudo apt install libreadline-dev gcc-arm-none-eabi libssl-dev cmake liblz4-dev libbz2-dev
\end{verbatim}

Following command clones the Proxmark3 repository.

\begin{Shaded}
\begin{Highlighting}[]
\FunctionTok{git}\NormalTok{ clone https://github.com/RfidResearchGroup/proxmark3.git}
\end{Highlighting}
\end{Shaded}

Next it is important to copy \texttt{Makefile.platform.sample} to
\texttt{Makefile.platform}, uncomment \texttt{PLATFORM=PM3GENERIC} and
comment out \texttt{PLATFORM=PM3RDV4}. This will set up the program for
Proxmark 3 Easy device.

Following script compiles and install the Proxmark3 software

\begin{Shaded}
\begin{Highlighting}[]
\FunctionTok{make}\NormalTok{ all }\KeywordTok{\&\&} \FunctionTok{make}\NormalTok{ install}
\end{Highlighting}
\end{Shaded}

\subsection{Flashing the Proxmark 3
Easy}\label{flashing-the-proxmark-3-easy}

\begin{quote}
Before proceeding, ensure to disable ModemManager on the utilized
system! More information avaiable here:
https://github.com/RfidResearchGroup/proxmark3/blob/master/doc/md/Installation\_Instructions/ModemManager-Must-Be-Discarded.md
\end{quote}

After successful installation of Proxmark3, flash the Proxmark 3 Easy
device by connecting the device to the computer and by running following
script.

\begin{verbatim}
sudo ./pm3-flash-all
\end{verbatim}

\subsection{Installing this software}\label{installing-this-software}

\begin{Shaded}
\begin{Highlighting}[]
\NormalTok{git clone git@gitlab.fit.cvut.cz:benesm41/final{-}project.git cloner}
\end{Highlighting}
\end{Shaded}

To install necessary files, run following commands

\begin{verbatim}
chmod +x install.sh
./install.sh
\end{verbatim}

And then edit the config file in \texttt{\textasciitilde{}/.cloner} for
your preferences. Default values are set.

After rebooting, the program starts on immediatelly and the device is
ready to use.
