\chapter{Conclusion}

% Cílem práce bylo sestrojit open-source zařízení, které bude schopné klonovat a emulovat RFID karty a tagy. Součástí práce mělo být navrhnout hardware a software architekturu přenosného zařízení  with respect to modularity and future extensibility za použitití nějakého commercially available microcomputer such as Raspberry Pi. Poté vytvoření programu umožňující klonování a emulaci alespoň 3 různých tagů, s debugováním přímo ve vytvořeném zařízení. S tím spojený i vývoj user interface který bude umožňovat jednoduché ovládání a interakci se zařízením. Poslední část práce měla zahrnovat otestování zařízení s různými typy RFID tagů a karet a vyhodnocení jeho funkčnosti.

% Byl vytvořen návrh prototypu přenosného a rožšířitelného zařízení využitím microcomputer Raspberry Pi a RFID nástroje Proxmark, který umožnujě klonování nebo emulaci množství těch nejznámějších typů RFID karet a tagů, jako je například Mifare Classic, Mifare Ultralight nebo Legic Prime. Byl vyvinut program v jazyce Python s grafickým rozhraním, který dokáže zařízení jednoduše a intuitivně ovládat.

% Vytvořený prototyp zařízení byl pečlivě otestován na zakoupených i již fungujících tagech a kartách. Vyšlo najevo, že testované karty a tagy dokáže bez větších problémů číst, klonovat i emulovat.

% V téhle fázi vývoje je zařízení pro dané typy RFID technologií funkční, avšak pro vylepšení některých vlastností budou zapotřebí další úpravy.

The aim of the work was to build an open-source device that would be able to clone and emulate RFID cards and tags. Part of the work was to design the hardware and software architecture of a portable device with respect to modularity and future extensibility using a commercially available microcomputer such as Raspberry Pi. Then creating a program allowing cloning and emulation of at least 3 different tags, with debugging directly in the created device. Associated with this is the development of a user interface that will allow easy control and interaction with the device. The last part of the work was to involve testing of the device with different types of RFID tags and cards and evaluating its functionality.

A prototype of a portable and scalable device was designed using a Raspberry Pi microcomputer and the RFID tool Proxmark, which allows cloning or emulation of a number of the most well-known RFID card and tag types such as Mifare Classic, Mifare Ultralight or Legic Prime. A Python program has been developed with a graphical interface that can operate the device in a simple and intuitive way.

The prototype device was carefully tested on purchased and existing tags and cards. It turned out that it can read, clone and emulate the tested cards and tags without major problems.

At this stage of development, the device is functional for the given types of RFID technology, but further modifications will be needed to improve some features.
