% Do not forget to include Introduction
%---------------------------------------------------------------
% \chapter{Introduction}
% uncomment the following line to create an unnumbered chapter
\chapter*{Introduction}\addcontentsline{toc}{chapter}{Introduction}\markboth{Introduction}{Introduction}
%---------------------------------------------------------------
\setcounter{page}{1}

% The following environment can be used as a mini-introduction for a chapter. Use that any way it pleases you (or comment it out). It can contain, for instance, a summary of the chapter. Or, there can be a quotation.
% \begin{chapterabstract}
% 	\lipsum[1]
% \end{chapterabstract}

Nowadays, the importance of RFID and NFC technologies has become increasingly crucial. People come across these technologies daily, validating their tickets in public transportation, accessing their offices, and paying with a card in a store. However, their significance extends beyond these everyday applications.

For instance, this innovation revolutionized the way businesses operate and interact with their environments. In retail, RFID tags have brought improvements in commodity management, allowing retailers to have a perfect overview of the stack levels in real time and optimize the supply to maximum efficiency. Logistic management benefits from RFID's ability to track the goods throughout the whole distribution process, reducing delays and eliminating certain management errors. In healthcare, NFC devices help with patient identification and medication administration, improving safety, accuracy, and the speed of healthcare processes. \cite{pirrone2012mobile} Similarly, the NFC technology is used in public transportation for ticketing, offering commuters a smart and convenient methods of payment and entry.

The history of RFID/NFC technologies can be traced back to World War II. The air forces of several countries were using radar, which was discovered in 1935 by physicist Sir Robert Alexander Watson-Watt, to warn of approaching airplanes. Germans discovered a method of identifying an enemy aircraft from allied aircraft by scanning the difference in the radio signal reflected by the airplane when the airplane rolled. This could be considered one of the earliest instances of passive RFID system in history. Later on, Sir Watson-Watt came up with the first active system, the so-called identification friend or foe system. Brits put an active transmitter on each airplane and when it received signals from the radars on the ground, it started broadcasting a signal back, identifying the airplane as friendly. Over the decades, significant advancements appeared. The first official active rewritable RFID tag was introduced in the 1970s by Mario W. Cardullo. That same year, Charles Walton received a patent for a passive transponder used to unlock doors. U.S. government had also an enormous influence on the development of this technology. Scientists from the Los Alamos National Laboratory developed a system to track nuclear materials in the area. Later, at the request of the Agricultural Department, Los Alamos also developed a system to track cows. Every cow has to be given hormones and medicines when it is ill and it was really hard to keep track of cows so that they do not receive two doses accidentally. \cite{violino2005history} As time went on, RFID and NFC technologies started to be used in more and more industries, as people realized the benefits it provides. These technologies continue to evolve rapidly, driven by innovations such as extended read ranges or enhanced data encryption. The adoption of NFC in smartphones has even further accelerated the integration into everyday life, enabling for example contactless payments or data transfer.

Many security systems heavily rely on RFID/NFC technologies for access control, authentication, or asset protection. The access systems that are commonly found in office buildings, hotels, or even houses, use RFID technology to grant or restrict entry based on authorized credentials. Asset tracking solutions use RFID tags to monitor the location or movement of valuable assets, reducing the risk of a theft.

In conclusion, the significance of RFID/NFC technologies extends into various industries, from retail and logistics to healthcare and transportation. RFID/NFC technologies play an important role as components of security systems, in ensuring the confidentiality and integrity of data, while also improving convenience and efficiency for users worldwide.
